% CVPR 2022 Paper Template
% based on the CVPR template provided by Ming-Ming Cheng (https://github.com/MCG-NKU/CVPR_Template)
% modified and extended by Stefan Roth (stefan.roth@NOSPAMtu-darmstadt.de)

\documentclass[10pt,twocolumn,letterpaper]{article}

%%%%%%%%% PAPER TYPE  - PLEASE UPDATE FOR FINAL VERSION
\usepackage{cvpr}      % To produce the REVIEW version
%\usepackage{cvpr}              % To produce the CAMERA-READY version
%\usepackage[pagenumbers]{cvpr} % To force page numbers, e.g. for an arXiv version

% Include other packages here, before hyperref.
\usepackage{graphicx}
\usepackage{amsmath}
\usepackage{amssymb}
\usepackage{booktabs}


% It is strongly recommended to use hyperref, especially for the review version.
% hyperref with option pagebackref eases the reviewers' job.
% Please disable hyperref *only* if you encounter grave issues, e.g. with the
% file validation for the camera-ready version.
%
% If you comment hyperref and then uncomment it, you should delete
% ReviewTempalte.aux before re-running LaTeX.
% (Or just hit 'q' on the first LaTeX run, let it finish, and you
%  should be clear).
\usepackage[pagebackref,breaklinks,colorlinks]{hyperref}


% Support for easy cross-referencing
\usepackage[capitalize]{cleveref}
\crefname{section}{Sec.}{Secs.}
\Crefname{section}{Section}{Sections}
\Crefname{table}{Table}{Tables}
\crefname{table}{Tab.}{Tabs.}


%%%%%%%%% PAPER ID  - PLEASE UPDATE
\def\cvprPaperID{*****} % *** Enter the CVPR Paper ID here
\def\confName{CVPR}
\def\confYear{2022}


\begin{document}

%%%%%%%%% TITLE - PLEASE UPDATE
\title{Fish Detection and Classification Through Deep Learning}

\author{Chan Tsz To, Jack\\
{\tt\small ttchanak@connect.ust.hk}
% For a paper whose authors are all at the same institution,
% omit the following lines up until the closing ``}''.
% Additional authors and addresses can be added with ``\and'',
% just like the second author.
% To save space, use either the email address or home page, not both
\and
Ko Sung Kit, Marcus\\
{\tt\small skkoab@connect.ust.hk}
\and
Wan Hanzhe\\
{\tt\small hwanad@connect.ust.hk}
}
\maketitle

%%%%%%%%% ABSTRACT
\begin{abstract}
The classification of fish species is an important task in the fishing industry. However, classifying fish manually is labor-intensive and prone to errors. This project proposes a machine learning-based classification model to automate fish species identification. By leveraging computer vision and a diverse dataset of fish images, the model aims to streamline fish classification, reduce costs, and improve accuracy.
\end{abstract}

%%%%%%%%% BODY TEXT
\section{Introduction}
According to a paper\cite{kay2021fishnetopenimagesdatabase}, over a thousand commercial fishing vessels are equipped with electronic monitoring (EM) systems, which utilize onboard cameras to monitor fishing activities and ensure accountability in the global seafood industry. These systems generate vast amounts of video data that are analyzed by trained human reviewers. In the coming decade, the number of vessels using electronic monitoring systems is projected to increase by 10 to 20 times, surpassing the current capacity for data review.

There are a lot of fish datasets that are of high quality and well-documented. We will train our model in different datasets to compare their performance. We have found the following datasets: Fishnet.AI, A Large Scale Fish Dataset, and Fish Dataset.

We would utilize some popular pre-trained models to facilitate the project. After that, we will compare the results and evaluate the performance of various models trained by different datasets. The main performance measure is the model's accuracy in correctly identifying fish species from images. Additionally, we will evaluate its accuracy in comparison to existing fish classification models.
%%%%%%%%% REFERENCES
{\small
\bibliographystyle{ieee_fullname}
\bibliography{egbib}
}

\end{document}
